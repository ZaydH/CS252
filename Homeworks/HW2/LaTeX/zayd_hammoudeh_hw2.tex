\documentclass{article}

\usepackage{fullpage}
\usepackage{listings}
\usepackage{amsmath}
\usepackage{amsthm}
\usepackage{amssymb}
%\usepackagen{url}
\usepackage{float}
\usepackage{paralist}

\floatstyle{boxed}
\restylefloat{figure}



\newcommand{\rel}[1]{ \mbox{\sc [#1]} }

\title{Homework 2: Operational Semantics for WHILE}

\author{
  CS 252: Advanced Programming Languages \\
  Prof. Thomas H. Austin \\
  San Jos\'{e} State University \\
  }
\date{}

\begin{document}
\maketitle

\section{Introduction}

For this assignment,
you will implement the semantics for a small imperative language, named WHILE.

% Commands for formatting figure
\newcommand{\mydefhead}[2]{\multicolumn{2}{l}{{#1}}&\mbox{\emph{#2}}\\}
\newcommand{\mydefcase}[2]{\qquad\qquad& #1 &\mbox{#2}\\}

% Commands for language format
\newcommand{\assign}[2]{#1~{:=}~#2}
\newcommand{\ife}[3]{\mbox{\tt if}~{#1}~\mbox{\tt then}~{#2}~\mbox{\tt else}~{#3}}
\newcommand{\whilee}[2]{\mbox{\tt while}~(#1)~#2}
\newcommand{\note}[1]{\mbox{\tt not}~#1}
\newcommand{\ande}[2]{\mbox{\tt and}~#1~#2}
\newcommand{\ore}[2]{\mbox{\tt or}~#1~#2}
\newcommand{\true}{\mbox{\tt true}}
\newcommand{\false}{\mbox{\tt false}}

\begin{figure}\label{fig:lang}
\caption{The WHILE language}
\[
\begin{array}{llr}
  \mydefhead{e ::=\qquad\qquad\qquad\qquad}{Expressions}
  \mydefcase{x}{variables/addresses}
  \mydefcase{v}{values}
  \mydefcase{\assign x e}{assignment}
  \mydefcase{e; e}{sequential expressions}
  \mydefcase{e ~op~ e}{binary operations}
  \mydefcase{\ife e e e}{conditional expressions}
  \mydefcase{\whilee e e}{while expressions}
  \mydefcase{\note e}{not expressions}
  \mydefcase{\ande e e}{and expressions}
  \mydefcase{\ore e e}{or expressions}
  \\
  \mydefhead{v ::=\qquad\qquad\qquad\qquad}{Values}
  \mydefcase{i}{integer values}
  \mydefcase{b}{boolean values}
  \\
  op ::= & + ~|~ - ~|~ * ~|~ / ~|~ > ~|~ >= ~|~ < ~|~ <=  & \mbox{\emph{Binary operators}} \\
\end{array}
\]
\end{figure}
($e_1;e_2$)



%---------
\section{Small-step semantics}

\newcommand{\ssrule}[3]{
  \rel{#1} &
  \frac{\strut\begin{array}{@{}c@{}} #2 \end{array}}
       {\strut\begin{array}{@{}c@{}} #3 \end{array}}
   \\~\\
}
\newcommand{\sstep}[4]{\ctxt[{#1}],{#2} \rightarrow \ctxt[{#3}],{#4}}
\newcommand{\sstepraw}[4]{{#1},{#2} \rightarrow {#3},{#4}}
\newcommand{\ctxt}{C}


Most of these rules are fairly straightforward, but there are a couple of points
to note with the $\rel{ss-while}$ rule.


\begin{figure}[H]\label{fig:smallstep}
\caption{Small-step semantics for WHILE}
{\bf Runtime Syntax:}
\[
\begin{array}{rclcl}
  \ctxt & \in & {Context} \quad & ::= & \quad \ctxt; e
        ~|~ \ctxt ~op~ e
        ~|~ v ~op~ \ctxt
        ~|~ \assign{x}{\ctxt}
        ~|~ \ife{\ctxt}{e_1}{e_2}
        ~|~ \bullet \\
  \sigma & \in & {Store} \quad  & = & \quad {variable} ~\rightarrow ~v \\
  \\
\end{array}
\]
{\bf Evaluation Rules:~~~ \fbox{$\sstepraw{e}{\sigma}{e'}{\sigma'}$}} \\
\[
\begin{array}{r@{\qquad\qquad}l}
\ssrule{ss-var}{
  x \in domain(\sigma) \qquad \sigma(x)=v
}{
  \sstep{x}{\sigma}{v}{\sigma}
}
\ssrule{ss-assign}{
}{
  \sstep{\assign{x}{v}}{\sigma}{v}{\sigma[x:=v]}
}
\ssrule{ss-op}{
  v = v_1 ~op~ v_2
}{
  \sstep{v_1~op~v_2}{\sigma}{v}{\sigma}
}
\ssrule{ss-seq}{
}{
  \sstep{v;e}{\sigma}{e}{\sigma}
}
\ssrule{ss-iftrue}{
}{
  \sstep{\ife{\true}{e_1}{e_2}}{\sigma}{e_1}{\sigma}
}
\ssrule{ss-iffalse}{
}{
  \sstep{\ife{\false}{e_1}{e_2}}{\sigma}{e_2}{\sigma}
}
\ssrule{ss-while}{
}{
  \sstep{\whilee{e_1}{e_2}}{\sigma}{\ife{e_1}{e_2;\whilee{e_1}{e_2}}{\false}}{\sigma}
}
\end{array}
\]
\end{figure}



\end{document}

