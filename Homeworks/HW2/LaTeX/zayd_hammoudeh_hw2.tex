\documentclass{article}

\usepackage{fullpage}
\usepackage{listings}
\usepackage{amsmath}
\usepackage{amsthm}
\usepackage{amssymb}
%\usepackagen{url}
\usepackage{float}
\usepackage{paralist}
\usepackage{indentfirst}
\usepackage{mdframed}

\floatstyle{boxed}
\restylefloat{figure}



\newcommand{\rel}[1]{ \mbox{\sc [#1]} }
% Commands for formatting figure
\newcommand{\mydefhead}[2]{\multicolumn{2}{l}{{#1}}&\mbox{\emph{#2}}\\}
\newcommand{\mydefcase}[2]{\qquad\qquad& #1 &\mbox{#2}\\}

% Commands for language format
\newcommand{\assign}[2]{#1~{:=}~#2}
\newcommand{\ife}[3]{\mbox{\tt if}~{#1}~\mbox{\tt then}~{#2}~\mbox{\tt else}~{#3}}
\newcommand{\whilee}[2]{\mbox{\tt while}~(#1)~#2}
\newcommand{\note}[1]{\mbox{\tt not}~#1}
\newcommand{\ande}[2]{\mbox{\tt and}~(#1)~(#2)}
\newcommand{\ore}[2]{\mbox{\tt or}~(#1)~(#2)}
\newcommand{\true}{\mbox{\tt true}}
\newcommand{\false}{\mbox{\tt false}}

\title{Homework \#2: Operational Semantics for the WHILE Language}

\author{
  Zayd Hammoudeh \\
  (zayd.hammoudeh@sjsu.edu)
  }
\date{}

\hfuzz=5.002pt  % Suppress hbox errors below 5pt.

\setlength\parskip{\baselineskip}

\begin{document}
\maketitle

\section{Introduction to the WHILE Language}

The ``WHILE" language is a basic language that was discussed in class.  Figure~\ref{fig:lang} defines the expressions, values, operators, and store in this language; the notation for expressions ($e$), values ($v$), variables/addresses ($x$), and store ($\sigma$) applies to all sections of this document.

\newcommand{\ssrule}[3]{
  \rel{#1} &
  \frac{\strut\begin{array}{@{}c@{}} #2 \end{array}}
       {\strut\begin{array}{@{}c@{}} #3 \end{array}}
   \\~\\
}
\newcommand{\sstep}[4]{{#1},{#2} \rightarrow {#3},{#4}}
\newcommand{\sstepraw}[4]{{#1},{#2} \rightarrow {#3},{#4}}
\begin{figure}[H]
\caption{The WHILE language}
\label{fig:lang}
\[
\begin{array}{llr}
  \mydefhead{e ::=\qquad\qquad\qquad\qquad}{Expressions}
  \mydefcase{x}{variables/addresses}
  \mydefcase{v}{values}
  \mydefcase{\assign x e}{assignment}
  \mydefcase{e; e}{sequential expressions}
  \mydefcase{e ~op~ e}{binary operations}
  \mydefcase{\ife e e e}{conditional expressions}
  \mydefcase{\whilee e e}{while expressions}
  \mydefcase{\note e}{not expressions}
  \mydefcase{\ande e e}{and expressions}
  \mydefcase{\ore e e}{or expressions}
  \\
  \mydefhead{v ::=\qquad\qquad\qquad\qquad}{Values}
  \mydefcase{i}{integer values}
  \mydefcase{b}{boolean values}
  \\
  op ::= & + ~|~ - ~|~ * ~|~ / ~|~ > ~|~ >= ~|~ < ~|~ <=  & \mbox{\emph{Binary operators}} 
  \\
  \\
  \mydefhead{\sigma}{Store} 
\end{array}
\]
\end{figure}

The store ($\sigma$) is a container that holds key value pairs of variables to values.  It is defined via the notation shown in figure~\ref{fig:sigma}.

\begin{figure}[H]
\caption{Store ($\sigma$) in the WHILE Language}
\label{fig:sigma}
\[
\\
\begin{array}{r@{\qquad\qquad}l}
  \sigma \in {Store} \quad  = \quad {variable} ~\rightarrow ~v
\end{array}
\]
\end{figure}

%---------
\section{Unextended WHILE Language Small-Step Semantics Rules}

In the WHILE language, the evaluation relation takes the form shown in figure \ref{fig:whileRelation}.

\begin{figure}[H]
\caption{WHILE Language Evaluation Relation}
\label{fig:whileRelation}
\[
\\
\begin{array}{r@{\qquad\qquad}l}
  \sstep{e}{\sigma}{e'}{\sigma'} \\
\end{array}
\]
\end{figure}

The following subsections enumerate the evaluation order, small-step semantics rules for the WHILE language expressions that were explicitly defined in class.  

\subsection{Variable Expression}

The variable expression is used to specify a key, which should correspond to a specific value in the store.  Note that it is possible for the key (i.e. variable) to not exist in the store.

\begin{figure}[H]
\caption{Variable Small-Step Semantics Evaluation Order Rule}
\label{fig:varRules}
{\bf Variable Evaluation Rule:}
\[
\\
\begin{array}{r@{\qquad\qquad}l}
\ssrule{ss-var}{
  x \in domain(\sigma) \qquad \sigma(x)=v
}{
  \sstep{x}{\sigma}{v}{\sigma}
}
\end{array}
\]
\end{figure}

\subsection{Set/Assign Expression}

The set/assignment expression is the complement of the variable expression.  In contrast to variable which extracts a value from the store, set/assign expression specifies a key (variable) and value, the pair of which is preserved in the store.

\begin{figure}[H]
\caption{Set/Assignment Small-Step Semantics Evaluation Order Rules}
\label{fig:assignRules}
{\bf Set/Assignment Evaluation Rules:} 
\[
\\
\begin{array}{r@{\qquad\qquad}l}
\ssrule{ss-assignContext}{
  \sstep{e}{\sigma}{e'}{\sigma'}
}{
  \sstep{\assign{x}{e}}{\sigma}{\assign{x}{e'}}{\sigma'}
}
\ssrule{ss-assignReduction}{
}{
  \sstep{\assign{x}{v}}{\sigma}{v}{\sigma[x:=v]}
}
\end{array}
\]
\end{figure}

\subsection{Binary Operator Expression}

By definition, binary operators take exactly two input parameters and return a result.  Figure~\ref{fig:opRules} defines the binary operator small-step evaluation order rules for the WHILE language.  Also, note that the supported set of binary operators is specified in figure~\ref{fig:lang}.

\begin{figure}[H]
\caption{Binary Operator ({\tt op}) Evaluation Order Rules}
\label{fig:opRules}
{\bf Binary Operator ({\tt op}) Evaluation Rules:}
\[
\\
\begin{array}{r@{\qquad\qquad}l}
\ssrule{ss-opContext1}{
  \sstep{e_1}{\sigma}{e_1'}{\sigma'}
}{
  \sstep{e_1~op~e_2}{\sigma}{e_1'~op~e_2}{\sigma'}
}
\ssrule{ss-opContext2}{
  \sstep{e}{\sigma}{e'}{\sigma'}
}{
  \sstep{v~op~e}{\sigma}{v~op~e'}{\sigma'}
}
\ssrule{ss-opReduction}{
  v_3 = v_1 ~op~ v_2
}{
  \sstep{v_1~op~v_2}{\sigma}{v_3}{\sigma}
}
\end{array}
\]
\end{figure}

\subsection{Sequence Expression}

The sequence expression is used when two or more distinct expressions need to be executed in a specific order; as such, it defines which of the set of specified expressions has precedence.  The rules in figure~\ref{fig:seqRules} are for two expressions, but this is extensible to an arbitrary number of expressions by chaining.

\begin{figure}[H]
\caption{Sequence ({\tt ;}) Evaluation Order Rules}
\label{fig:seqRules}
{\bf Sequence ({\tt ;}) Evaluation Rules:}
\[
\\
\begin{array}{r@{\qquad\qquad}l}
\ssrule{ss-seqContext}{
  \sstep{e_1}{\sigma}{e_1'}{\sigma'}
}{
  \sstep{e_1 ; e_2}{\sigma}{e_1' ; e_2}{\sigma'}
}
\ssrule{ss-seqReduction}{
}{
  \sstep{v ; e}{\sigma}{e}{\sigma}
}
\end{array}
\]
\end{figure}

\subsection{Conditional ({\tt if}) Expression}

The conditional ({\tt if}) in the WHILE language is similar to that of other languages in that it takes a Boolean value (e.g. \true or \false) and returns one of two possible results depending on that Boolean value.  The rules for {\tt if} are in figure~\ref{fig:condRules}.

\begin{figure}[H]
\caption{Conditional ({\tt if}) Small-Step Semantics Evaluation Order Rules}
\label{fig:condRules}
{\bf Conditional Statement ({\tt if}) Evaluation Rules:}
\[
\\
\begin{array}{r@{\qquad\qquad}l}
\ssrule{ss-ifContext}{
  \sstep{e_1}{\sigma}{e_1'}{\sigma'}
}{
  \sstep{\ife{e_1}{e_2}{e_3}}{\sigma}{\ife{e_1'}{e_2}{e_3}}{\sigma'}
}
\ssrule{ss-ifTrueReduction}{
}{
  \sstep{\ife{\true}{e_1}{e_2}}{\sigma}{e_1}{\sigma}
}
\ssrule{ss-ifFalseReduction}{
}{
  \sstep{\ife{\false}{e_1}{e_2}}{\sigma}{e_2}{\sigma}
}
\end{array}
\]
\end{figure}

\subsection{While Expression}

Similar to the {\tt while} construct in other programming languages, the {\tt while} expression in this language takes two expressions and will evaluate the second expression as long as the first expression evaluates to \true.

\begin{figure}[H]
\caption{{\tt while} Small-Step Semantics Evaluation Order Rule}
\label{fig:whileRules}
{\bf {\tt while} Evaluation Rule:} 
\[
\\
\begin{array}{r@{\qquad\qquad}l}
\ssrule{ss-whileReduction}{
}{
  \sstep{\whilee{e_1}{e_2}}{\sigma}{\ife{e_1}{(e_2;\whilee{e_1}{e_2})}{\false}}{\sigma}
}
\end{array}
\]
\end{figure}

%---------
\section{Boolean Expression Small-Step Semantics Rules for an Extended WHILE Language}


In following subsections, I describe three additional expression types in the updated/extended WHILE language namely: {\tt not}, {\tt and}, and {\tt or}.  

\subsection{{\tt not} Expression}\label{subSec:not}

{\tt not} in my modified version of the WHILE language behaves as a standard Boolean {\tt not}.  It takes a single Boolean value and returns its complement.  If an expression is passed, the language simplifies that expression until it is in normal form at which point it applies the Boolean {\tt not}.

\begin{figure}[H]
\caption{{\tt not} Small-Step Semantics Evaluation Order Rule}
\label{fig:notRules}
{\bf {\tt not} Evaluation Rule:} 
\[
\\
\begin{array}{r@{\qquad\qquad}l}
\ssrule{ss-notReduction}{
}{
  \sstep{\note{e}}{\sigma}{\ife{e}{\false}{\true}}{\sigma}
}
\end{array}
\]
\end{figure}

\subsection{{\tt and} Expression}\label{subSec:and}

{\tt and} is designed to mimic the Boolean {\tt and} with the exception that it supports short circuit compare.  Hence, if the first expression in the {\tt and} evaluates to {\false}, the second parameter is not evaluated at all.

\begin{figure}[H]
\caption{{\tt and} Small-Step Semantics Evaluation Order Rules}
\label{fig:andRules}
{\bf {\tt and} Evaluation Rules:}
\[
\\
\begin{array}{r@{\qquad\qquad}l}
\ssrule{ss-andContext}{
  \sstep{e_1}{\sigma}{e_1'}{\sigma'}
}{
  \sstep{\ande{e_1}{e_2}}{\sigma}{\ande{e_1'}{e_2}}{\sigma'}
}
\ssrule{ss-andReduction}{
e'=\ife{e}{\true}{\false}
}{
  \sstep{\ande{v}{e}}{\sigma}{\ife{v}{e'}{\false}}{\sigma}
}
\end{array}
\]
\end{figure}

\subsection{{\tt or} Expression}

{\tt or} is a composite of the expressions ``{\tt not}" and ``{\tt and}" described in sections \ref{subSec:not} and \ref{subSec:and} respectively.

\begin{figure}[H]
\caption{{\tt or} Small-Step Semantics Evaluation Order Rule}
\label{fig:orRules}
{\bf {\tt or} Evaluation Rule:} 
\[
\\
\begin{array}{r@{\qquad\qquad}l}
\ssrule{ss-orReduction}{
  e_1'=\note{e_1} \qquad e_2'=\note{e_2} \qquad e_3=\ande{e_1'}{e_2'}
}{
  \sstep{\ore{e_1}{e_2}}{\sigma}{\note{e_3}}{\sigma}
}
\end{array}
\]
\end{figure}

\end{document}

