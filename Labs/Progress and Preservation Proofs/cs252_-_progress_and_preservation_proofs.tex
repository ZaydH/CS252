\documentclass{report}

\usepackage{url}
\usepackage{indentfirst}
\usepackage{float}
\usepackage{listings} % Include the listings-package
\usepackage{mdframed}
\usepackage[numbers,sort]{natbib}
%\usepackage[english]{babel} % Need for text wrap in table.
\usepackage{array} % Needed for centering in the table
\usepackage{graphicx}

\usepackage{fullpage}
\usepackage{listings}
\usepackage{amsmath}
\usepackage{amsthm}
\usepackage{amssymb}
%\usepackagen{url}
\usepackage{float}
\usepackage{paralist}

\floatstyle{boxed}
\restylefloat{figure}

\usepackage{framed, color}
\definecolor{shadecolor}{rgb}{1, 0.8, 0.3}

\usepackage{hyperref} % Creates links in the PDF document.
\hypersetup{hidelinks} % Do not include boxes around links


\newcommand{\rel}[1]{ \mbox{\sc [#1]} }

\title{Progress and Preservation Proofs for the \\
Expressions ``\texttt{iszero}" and ``\texttt{pred}" in the Arith Language}

\author{
  Zayd Hammoudeh \\
  (zayd.hammoudeh@sjsu.edu)
  }


% Skip lines after each paragraph.
\setlength\parskip{\baselineskip}

\begin{document}

\maketitle

\pagenumbering{roman}

\tableofcontents{\protect\newpage}

\addcontentsline{toc}{section}{List of Figures}
\listoffigures
\newpage
 
\pagenumbering{arabic}

\renewcommand\thesection{\arabic{section}}

\section{Arith Language}\label{sec:jvm}

Arith is a basic language; its expressions, values, and types are enumerated in figure~\ref{fig:arithLanguage}.  Arith's small-step, evaluation order semantics are defined in figure~\ref{fig:arithSmallStepRules} while Arith's type rules are enumerated in figure~\ref{fig:arithTypeRules}.

% Commands for formatting figure
\newcommand{\mydefhead}[2]{\multicolumn{2}{l}{{#1}}&\mbox{\emph{#2}}\\}
\newcommand{\mydefcase}[2]{\qquad\qquad& #1 &\mbox{#2}\\}

% Commands for language format
\newcommand{\assign}[2]{#1~{:=}~#2}
\newcommand{\ife}[3]{\mbox{\tt if}~({#1})~\mbox{\tt then}~({#2})~\mbox{\tt else}~({#3})}
\newcommand{\iszeroe}[1]{\mbox{\tt iszero}~(#1)}
\newcommand{\prede}[1]{\mbox{\tt pred}~(#1)}
\newcommand{\succe}[1]{\mbox{\tt succ}~(#1)}
\newcommand{\true}{\mbox{\tt true}}
\newcommand{\false}{\mbox{\tt false}}
\newcommand{\zero}{\mbox{\tt 0}}
\newcommand{\boolt}{\mbox{\tt Bool}}
\newcommand{\intt}{\mbox{\tt Int}}

\begin{figure}[ht!]
\[
\begin{array}{llr}
  \mydefhead{e ::=\qquad\qquad\qquad\qquad}{Expressions}
  \mydefcase{\true}{Boolean True}
  \mydefcase{\false}{Boolean False}
  \mydefcase{\zero}{Integer Value 0}
  \mydefcase{\succe e}{Successor Expressions}
  \mydefcase{\prede e}{Predecessor Expressions}
  \mydefcase{\iszeroe e}{Zero Value Check Expressions}
  \mydefcase{\ife e e e}{Conditional Expressions}
  \\
  \mydefhead{v ::=\qquad\qquad\qquad\qquad}{Values}
  \mydefcase{i}{integer values}
  \mydefcase{b}{boolean values}
  \\
  \mydefhead{T ::=\qquad\qquad\qquad\qquad}{Types}
  \mydefcase{\boolt}{Boolean Type}
  \mydefcase{\intt}{Integer Type}
\end{array}
\]
\caption{The Arith language}\label{fig:arithLanguage}
\end{figure}


\newcommand{\ruleNoBar}[2]{
  \rel{#1} &
     \strut\begin{array}{@{}c@{}} #2 \end{array}
   \\~\\~\\
}
\newcommand{\ruleWithBar}[3]{
  \rel{#1} &
  \frac{\strut\begin{array}{@{}c@{}} #2 \end{array}}
       {\strut\begin{array}{@{}c@{}} #3 \end{array}}
   \\~\\~\\
}
\newcommand{\sstep}[2]{{#1} \rightarrow {#2}}
\newcommand{\typeRule}[2]{{#1} : {#2}}


\begin{figure}[H]
{\bf Evaluation Rules:~~~ \fbox{$\sstep{e}{e'}$}} \\
\[
\begin{array}{r@{\qquad\qquad}l}
\ruleWithBar{E-Succ-Ctxt}{
  \sstep{e_1}{e'_1}
}{
  \sstep{\succe {e_1}}{\succe {e'_1}}
}
\ruleWithBar{E-Succ}{
  i' = i + 1
}{
  \sstep{\succe i}{i'}
}
\ruleWithBar{E-Pred-Ctxt}{
  \sstep{e_1}{e'_1}
}{
  \sstep{\prede {e_1}}{\prede {e'_1}}
}
\ruleWithBar{E-Pred}{
  i' = i - 1
}{
  \sstep{\prede i}{i'}
}
\ruleWithBar{E-IsZero-Ctxt}{
  \sstep{e_1}{e'_1}
}{
  \sstep{\iszeroe {e_1}}{\iszeroe {e_1'}}
}
\ruleNoBar{E-IsZero-Z}{
  \sstep{\iszeroe \zero}{\true}
}
\ruleWithBar{E-IsZero-NZ}{
   i \ne 0
}{
  \sstep{\iszeroe i}{\false}
}
\ruleWithBar{E-If-Ctxt}{
  \sstep{e_1}{e'_1}
}{
  \sstep{\ife {e_1} {e_2} {e_3}}{\ife {e'_1} {e_2} {e_3}}
}
\ruleNoBar{E-If-True}{
  \sstep{\ife \true {e_2} {e_3}}{e_2}
}
\ruleNoBar{E-If-False}{
  \sstep{\ife \false {e_2} {e_3}}{e_3}
}
\end{array}
\]
\caption{Small-Step, Evaluation Order Semantics Semantics for the Arith Language}\label{fig:arithSmallStepRules}
\end{figure}

\begin{figure}[H]
{\bf Type Rules:~~~ \fbox{$\typeRule{e}{T}$}} \\
\[
\begin{array}{r@{\qquad\qquad}l}
\ruleNoBar{T-True}{
  \typeRule{\true}{\boolt}
}
\ruleNoBar{T-False}{
  \typeRule{\false}{\boolt}
}
\ruleNoBar{T-Int}{
  \typeRule{i}{\intt}
}
\ruleWithBar{T-Succ}{
  \typeRule{e_1}{\intt}
}{
  \typeRule{\succe e_1}{\intt}
}
\ruleWithBar{T-Pred}{
  \typeRule{e_1}{\intt}
}{
  \typeRule{\prede e_1}{\intt}
}
\ruleWithBar{T-IsZero}{
  \typeRule{e_1}{\intt}
}{
  \typeRule{\iszeroe e_1}{\boolt}
}
\ruleWithBar{T-If}{
  \typeRule{e_1}{\boolt},~~\typeRule{e_2}{T},~~\typeRule{e_3}{T}
}{
  \typeRule{\ife {e_1} {e_2} {e_3}}{T}
}
\end{array}
\]
\caption{Type Rules for the Arith Language}\label{fig:arithTypeRules}
\end{figure}



\section{Progress}\label{sec:progress}

In semantics context, ``progress" entails that a well-type expression will not ``get stuck."  Figure~\ref{fig:progressTheorem} is the formal, theoretical definition of progress.

\begin{figure}[H]

	Given $e : T$, then either:
	
	\begin{enumerate}
	
      \item $e$ is a value.
      
      \item There exists an $e'$ such that: $e \rightarrow e'$.
	
	\end{enumerate}
	
  \caption{Formal Definition of the Progress Theorem}\label{fig:progressTheorem}
\end{figure}

The following subsections are the formal proofs of progress for the type rules in figure~\ref{fig:arithTypeRules}.

\subsection{Proving Progress for Boolean and Integer Values}

Figure~\ref{fig:progressTheorem} establish that progress is achieved if an expression $e$ is a value.  Hence, by this criterion, type rules \rel{T-True}, \rel{T-False}, and \rel{T-Int} all hold.

\subsection{Proving Progress for the If Expression}

Figure~\ref{fig:ifProofProgress} is the proof of progress for the If expression in the Arith Language.

\begin{figure}[ht!]
Given:
\begin{center}
   $e = \ife {e_1} {e_2} {e_3}$
   $\typeRule{e_1}{\boolt},~~\typeRule{e_2}{T},~~\typeRule{e_3}{T}$
\end{center}
Then:
\begin{center}
   $\typeRule{e}{T}$
\end{center}
By induction, an expression $e_1$ must be a value or $\exists e_1$ such that:
\begin{enumerate}

    \item If $e_1 = i$ where ``$i$", then $\sstep{e}{i'}$ where:
    
    \begin{center}
       $i' = i + 1$
    \end{center}
      
    \item Otherwise, $\sstep{e_1}{e'_1}$ which means that $\rel{E-If-Ctxt}$ applies (as shown below) which by induction holds.
    
    \begin{center}
      $\sstep{\succ e_1}{\ife {e'_1} {e_2} {e_3}}$
    \end{center}
	
	\end{enumerate}
	
  \caption{Proof of Progress for the If Expression}\label{fig:ifProofProgress}
\end{figure}

\subsection{Proving Progress for the Successor Expression}

Figure~\ref{fig:arithTypeRules} shows the type rule for the successor expression.  The proof of progress for this expression is shown in figure~\ref{fig:succProofProgress}.

\begin{figure}[ht!]
Given:
\begin{center}
  $e = \succe {e_1}$ \\
   $\typeRule{e_1}{\intt}$
\end{center}
Then:~\\
\begin{center}
   $T = \intt$
\end{center}

By induction, $e_1$ must either be a value or $\exists e'_1$ such that:~\\

	\begin{enumerate}
    \item If $e_1$ is a value, then either \rel{E-If-True} or \rel{E-If-False} applies since $\typeRule{e_1}{\boolt}$.
      
    \item Otherwise $\rel{E-If-Ctxt}$ applies (as shown below) which by induction holds.
    
    \begin{center}
      $\sstep{e}{\ife {e'_1} {e_2} {e_3}}$
    \end{center}
	
	\end{enumerate}
	
  \caption{Proof of Progress for the Successor Expression}\label{fig:succProofProgress}
\end{figure}

\end{document}